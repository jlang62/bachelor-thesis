\chapter{Introduction}

The classification of text documents is essential across various applications. It demands a high accuracy and efficiency, in order to be beneficial. Traditional approaches often struggle with the diversity and complexity of modern textual data, emphasizing the need for advanced techniques. Leveraging \ac{nlp}, data mining, and \ac{ml} techniques hold promise in overcoming these challenges by enabling automated categorization of textual documents based on their content, context, and semantics. These procedures contain numerous obstacles, including defining accurate annotation of documents, the implementation of dimensionality reduction techniques to address complexities, and utilizing appropriate classifier functions to obtain robust generalization while avoiding overfitting \citep{aurangzeb_review_2010}.

One of the main sources for textual documents these days is the internet, often known as the World Wide Web, the amount that is available to us is constantly increasing. Unstructured textual formats, such as reports, emails, opinions, and news stories, are thought to contain approximately 80\% or more of an organization's information. Studies indicate that unstructured formats contain almost 90\% of the world's data. There's a clear necessity for the automatic extraction of valuable insights from vast amounts of textual data to aid human analysis \citep{aurangzeb_review_2010}.

This study aims to investigate how these advanced classification techniques, can enhance the accuracy and efficiency of categorizing diverse textual documents. Libraries, such as scikit-learn, NLTK, and Keras, and their broad spectrum of algorithms will be used to implement different approaches and explore the interactions between these techniques. With the use of a BBC text document dataset with a “.csv” format, the research seeks to gain insights into optimizing classification performance in real-world applications.

The following research question arises from the objective: “How can advanced text document classification techniques, incorporating machine learning and natural language processing, be effectively employed to improve the accuracy and efficiency of categorizing diverse textual documents, and what factors influence the performance of such classification models in real-world applications?”

\section{Motivation}

The current state of research in text document classification techniques, with ML and NLP, showcases significant progress toward enhancing efficiency and accuracy. Recent developments have seen the emergence of sophisticated deep learning architectures, such as Recurrent Neural Networks (RNN) and Convolutional Neural Networks (CNN), which excel in processing complex textual data. Furthermore, has the integration of pre-trained language models like \ac{gpt} and \ac{bert} revolutionized feature representation, allowing models to capture difficult semantic nuances. This dynamic landscape reflects ongoing efforts to refine classification models, making them increasingly adept at real-world applications.
% \fig{img/sax_approximated_series}{Sax approximation of a time series}{fig:sax}{0.5}


% \section{Objectives}

% \blindtext

% \section{Methods}

% \blindtext

% \section{Structure}

% \blindtext

% \section{Tables}

% Table~\ref{tab:table-one} shows an example table.

% \begin{table}[htbp]
%     \centering
%     \caption{This is a table}
%     \label{tab:table-one}
%     \begin{tabular}{lll}
%         \addlinespace
%         \toprule
%         Column 1 & Column 2 & Column 3 \\
%         \midrule
%         A     & B     & C \\
%         D     & E     & F \\
%         G     & H     & I \\
%         \bottomrule
%     \end{tabular}
% \end{table}

% \section{Source Code}

% \begin{lstlisting}[language=Java, caption=Hello World in Java, label=lst:hello-world-java]
% public class Hello {
%     public static void main(String[] args) {
%         System.out.println("Hello World");
%     }
% }
% \end{lstlisting}

% Listing~\ref{lst:hello-world-java} shows the classic Hello World in Java.

% \lstinputlisting[language=Python, caption=Hello World in Python, label=lst:hello-world-py]{./lst/hello.py}

% Listing~\ref{lst:hello-world-py} shows the classic Hello World in Python.

% \begin{lstlisting}[language=JavaScript, caption=Hello World in JavaScript, label=lst:hello-world-javascript]
% function hello() {
%     console.log('Hello World');
% }

% hello();
% \end{lstlisting}

% \newpage
% \begin{lstlisting}[language=ES6, caption=Hello World in JavaScript (ES6), label=lst:hello-world-javascript]
% const hello = async () => {
%     await console.log('Hello World');
% }

% hello();
% \end{lstlisting}

% \section{Acronyms}
% It is also possible to define abbreviations, such as \ac{html} and \ac{js}. We use the \texttt{acronym} packages that provides several options like plurals \acp{js} (see \url{ftp://ftp.tu-chemnitz.de/pub/tex/macros/latex/contrib/acronym/acronym.pdf}).
