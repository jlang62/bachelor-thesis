\chapter{Theory}

The theoretical basis of this thesis provides the necessary background to comprehend the research discussed in the following chapters. Through an insight of relevant development in the field, various models, preprocessing methodologies, and evaluation metrics, this chapter aims to construct a theoretical framework for the thesis. Its objective is to empower the reader with the requisite understanding to contextualize the research outcomes and their significance.

\section{Natural Language Processing}

Natural Language Processing (NLP) is a branch of computer science and artificial intelligence, that deals with the interaction between human language and computers. Its aim is to enable machines to generate human language, process and understand it. Through employing a variety of techniques and different approaches, such as deep learning, rule-based systems, and statistical methods, is NLP capable to tackle different language-related tasks. The usage of it can be found in numerous areas, including machine translation, chatbots, text classification, and speech recognition (Helland, 2023).

\section{Deep Learning}

\Blindtext[4][1]
