\chapter{Theory}

This chapter attempts to build a theoretical foundation for the thesis by providing an understanding of significant developments in the field, different models, preprocessing techniques, and assessment measures. Its goal is to provide the reader with the necessary knowledge to interpret the research findings and their importance.

\section{Natural Language Processing}

In the domains of artificial intelligence and computer science, the study of the interaction between computers and human language is referred to as \acf{nlp} \citep{helland_tackling_2023}. It seeks to make it possible for machines to produce, interpret, and process human language. By employing a variety of techniques and different approaches, such as deep learning, rule-based systems, and statistical methods, is NLP capable of tackling different language-related tasks. The usage of it can be found in numerous areas, including machine translation, chatbots, text classification, and speech recognition \citep{helland_tackling_2023}.

\section{Machine Learning}

The term "machine learning" describes the creation of computer programmes that use past performance to solve problems and finish tasks. The measurement of the performance is calculated by the ability to do so \citep{helland_tackling_2023}. A training dataset represents the “experience” that is acquired by machine learning models, which contains output and input pairs. FThrough the examination of these instances, the model is able to identify and extrapolate the patterns to new, unobserved data \citep{helland_tackling_2023}. Simplifying, it can be understood as a reflection of how people learn and adjust to new situations. Self-driving cars, fraud detection, and personalized suggestions are examples of common applications \citep{helland_tackling_2023}.

\section{Deep Learning}

Deep learning is a technique that enables models to identify patterns in the raw data by transforming it at several levels of abstraction using non-linear modules \citep{helland_tackling_2023}. According to this, deep learning algorithms may be able to identify and acquire new data features without requiring human assistance \citep{helland_tackling_2023}. Because of its universal learning, generalization potential, robustness, and scalability advantages, this can be used in a variety of applications without the need for precise feature engineering \citep{helland_tackling_2023}.

\section{Evaluation Metrics}

A machine learning model's performance is assessed using metrics. They are employed to evaluate the accuracy of the made predictions, to analyze the output of various models, and to fine-tune them for optimal performance \citep{helland_tackling_2023}. Different types of machine learning issues have various types of metrics available. The model selection process, the optimization procedure, and the overall understanding of the model's capabilities can all be impacted by the metrics chosen. Incorrect metric selection may also lead to a biased model that is at contrast with the goals of the project \citep{helland_tackling_2023}.

\subsection{Accuracy}

One popular assessment metric for classification problems is “accuracy”. Out of all the samples in the prediction, it calculates the proportion of correctly classified samples \citep{helland_tackling_2023}. 

Accuracy = Number of correctly classified instances / Total number of instances

Accuracy is useful for balanced datasets since it gives a true representation of the model's performance and capabilities \citep{helland_tackling_2023}. When datasets are unbalanced, accuracy might be misleading and more difficult to interpret \citep{helland_tackling_2023}. This could be because there is only one label in the dataset that applies to most of the samples. In this case, projecting all samples to the dominant label will still yield a reasonable level of accuracy. This does not imply that the model is good because it ignores the less common but no less significant labels. Accuracy in multi-label classification only takes into account samples where every label is correctly classified. Because of this, using accuracy as a multi-label classification metric to evaluate the performance of multi-label models is more strict, less informative, and less desirable \citep{helland_tackling_2023}.